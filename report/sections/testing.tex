\section{Тестирование}

\setlength{\parskip}{1.0ex}
\small
\renewcommand{\arraystretch}{2}
\newcommand{\tab}{\hspace{0.9cm}}
\newcommand{\test}[1]{\begin{spacing}{0.5}\texttt{\begin{tabular}[c]{l}#1\end{tabular}}\end{spacing}}
\newcommand{\img}[1]{\hspace{0.2cm}\texttt{graph:}\begin{center}\includegraphics[scale=0.5]{graphics/#1}\end{center}}


\begin{xltabular}[h]{\textwidth}{|p{0.25 \textwidth}|X|X|}
    \caption{Тестирование конструктора\label{tab:constructor-testing}} \\
    \hline
    \textbf{Тестовая ситуация} & \textbf{Входные данные} & \textbf{Выходные данные} \\
    \hline \endhead
    Создание пустого графа & \test{Graph graph} & \test{graph: \{\\\tab head = nullptr;\\\tab tail = nullptr;\\\}} \\
    \hline
\end{xltabular}


\begin{xltabular}[h]{\textwidth}{|p{0.25 \textwidth}|X|X|}
    \caption{Тестирование деструктора\label{tab:destructor-testing}} \\
    \hline
    \textbf{Тестовая ситуация} & \textbf{Входные данные} & \textbf{Выходные данные} \\
    \hline \endhead
    Удаление пустого графа & \test{graph: \{\\\tab head = nullptr;\\\tab tail = nullptr;\\\}} & \test{graph: \{\\\tab head = nullptr;\\\tab tail = nullptr;\\\}} \\
    \hline
    Удаление непустого графа, где \texttt{head} и \texttt{tail} указывают на разные участки памяти & \test{graph: \{\\\tab head = 0x00000001;\\\tab tail = 0x00000002;\\\}} & \test{graph: \{\\\tab head = nullptr;\\\tab tail = nullptr;\\\}} \\
    \hline
    Удаление непустого графа, где \texttt{head} и \texttt{tail} указывают на один и тот же участок памяти & \test{graph: \{\\\tab head = 0x00000001;\\\tab tail = 0x00000001;\\\}} & \test{graph: \{\\\tab head = nullptr;\\\tab tail = nullptr;\\\}} \\
    \hline
\end{xltabular}


\begin{xltabular}[h]{\textwidth}{|p{0.25 \textwidth}|X|X|}
    \caption{Тестирование добавления дуги\label{tab:addarc-testing}} \\
    \hline
    \textbf{Тестовая ситуация} & \textbf{Входные данные} & \textbf{Выходные данные} \\
    \hline \endhead
    Добавление в пустой граф & \test{graph: \{\\\tab head = nullptr;\\\tab tail = nullptr;\\\}\\arc: \{1, 2\}} & \img{1}\test{return: true} \\
    \hline
    Добавление в непустой граф дуги, соединяющейся с уже существующей вершиной & \img{1}\test{arc: \{1, 3\}} & \img{2}\test{return: true} \\
    \hline
    Добавление в непустой граф дуги, не соединяющейся с уже существующей вершиной & \img{1}\test{arc: \{3, 4\}} & \img{3}\test{return: true} \\
    \hline
    Добавление петли в пустой граф & \test{graph: \{\\\tab head = nullptr;\\\tab tail = nullptr;\\\}\\arc: \{1, 1\}} & \img{4}\test{return: true} \\
    \hline
    Добавление петли в уже существующую вершину & \img{1}\test{arc: \{1, 1\}} & \img{5}\test{return: true} \\
    \hline
    Добавление уже существующей дуги & \img{1}\test{arc: \{1, 2\}} & \img{6}\test{return: true} \\
    \hline
    Добавление петли, не выходящей ни из одной уже существующей вершины & \img{1}\test{arc: \{3, 3\}} & \img{7}\test{return: true} \\
    \hline
\end{xltabular}


\begin{xltabular}[h]{\textwidth}{|p{0.25 \textwidth}|X|X|}
    \caption{Тестирование удаления дуги\label{tab:deletearc-testing}} \\
    \hline
    \textbf{Тестовая ситуация} & \textbf{Входные данные} & \textbf{Выходные данные} \\
    \hline \endhead
    Удаление из пустого графа & \test{graph: \{\\\tab head = nullptr;\\\tab tail = nullptr;\\\}\\arc: \{1, 2\}} & \test{graph: \{\\\tab head = nullptr;\\\tab tail = nullptr;\\\}\\return: false} \\
    \hline
    Удаление не существующей дуги из непустого графа & \img{1}\test{arc: \{1, 3\}} & \img{1}\test{return: false} \\
    \hline
    Удаление существующей дуги & \img{2}\test{arc: \{1, 3\}} & \img{1}\test{return: true} \\
    \hline
    Удаление последней дуги & \img{1}\test{\\arc: \{1, 2\}} & \test{graph: \{\\\tab head = nullptr;\\\tab tail = nullptr;\\\}\\return: true} \\
    \hline
\end{xltabular}


\newpage